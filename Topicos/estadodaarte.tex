Atualmente, a tecnologia tem desempenhado um papel fundamental no combate às doenças. A eficiência na análise de dados e a rápida troca de informações são essenciais para minimizar o número de vítimas durante crises de saúde. Nesta seção, apresentaremos alguns exemplos desses avanços e discutiremos como o Software EPIDEMIX se destaca em relação a outras tecnologias.

No estudo realizado T. et al. (2022), foram utilizadas informações médicas de pacientes com hanseníase fornecidas pelos Centros de Saúde Primária (PHCs) das cidades de Pamekasan e Pasuruan, na Indonésia. Em colaboração com uma equipe de pesquisadores e voluntários, Taal realizou visitas domiciliares aos pacientes para coletar suas coordenadas geográficas (latitude e longitude) utilizando a aplicação MapIt, que usa o Geographic Positioning Systems (GPS) para a coleta dos dados. Essas coordenadas foram processadas pelo sistema de código-aberto Quantum Geographic Information System (QGIS), e os agrupamentos de casos de hanseníase foram calculados utilizando o programa ClusterSeer. Em seguida, a ferramenta de mapa de calor do QGIS foi empregada para gerar uma representação visual da distribuição dos casos com base em diferentes níveis de densidade.

O estudo identificou um grande agrupamento de casos em Pamekasan e Pasuruan, com quase 100\% dos casos localizados dentro das áreas destacadas no mapa de calor. Estima-se que entre 21\% e 90\% dos casos futuros estarão dentro das áreas identificadas. Essas informações podem ser usadas para melhorar a eficácia das estratégias de combate e prevenção dessas doenças.

Em Miguel, Hornink e Bressan (2020), Miguel desenvolve um aplicativo móvel destinado a identificar e visualizar áreas de foco para Dengue, Zika e Chikungunya em um mapa, seja através de pontos individuais ou mapas de calor. Os usuários podem inserir um registro de foco a qualquer momento, simplesmente pressionando no mapa na tela inicial do aplicativo. As informações sobre a localização dos focos são armazenadas em um banco de dados, processadas e, em seguida, exibidas na interface do mapa do aplicativo, que também oferece a opção de filtrar por tipo de registro ou por período. No entanto, os dados utilizados para demonstrar o aplicativo eram fictícios, deixando uma lacuna na validação prática do software em situações reais. Embora a ferramenta tenha demonstrado eficácia teórica, não foram consideradas a possibilidade de registros falsos pelos usuários ou registros repetidos, já que mais de uma pessoa poderia registrar o mesmo local, resultando na criação de uma grande área de risco falsa. Ambas as possibilidades podem impactar negativamente a precisão das informações apresentadas.



