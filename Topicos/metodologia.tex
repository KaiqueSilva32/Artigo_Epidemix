O estudo propõe um sistema dedicado à identificação de áreas de risco de dengue no Brasil. Seu propósito é fornecer suporte aos agentes de saúde para que a dedetização dos locais aconteça de forma eficiente. A fim de que o sistema seja satisfatório, algumas técnicas que serão utilizadas são cruciais, como suavizar os pontos afetados em um HeatMap para facilitar a visualização, além de fornecer rotas eficazes aos agentes de dedetização para que mais regiões sejam tratadas em menos tempo, auxiliando no combate à proliferação dessa doença. A metodologia deste estudo envolve o uso de IA para que as técnicas citadas possam ser aplicadas ao projeto, assim como ferramentas de design e desenvolvimento de software. 

Para a criação da interface do usuário do software EPIDEMIX, empregamos o Figma. Este software de design gráfico permite a criação de protótipos interativos de baixa e alta fidelidade, com layouts e estilos, facilitando a elaboração do design de forma clara aos usuários e permitindo modificações conforme necessário.

Além disso, utilizamos a plataforma SEBRAE Canvas para planejar o modelo de negócios do software. Esta plataforma fornece um quadro dividido em seções como parcerias, proposta de valor, estrutura de custos e fontes de receita, auxiliando no planejamento de cada aspecto do modelo de negócios e na definição dos elementos-chave do projeto. É importante citar que o Canvas do projeto foi desenvolvido antes mesmo do aplicativo, pois a partir dele uma visão lógica e completa do sistema pode ser concebida.

Com o auxílio do brModelo, criamos o modelo conceitual e lógico do banco de dados necessário para o funcionamento do EPIDEMIX. O brModelo é um software que facilita a criação dos modelos de um banco de dados usando diagramas de entidade-relacionamento (DER). Com os modelos finalizados, é possível desenvolver a interface do aplicativo, garantindo uma compreensão detalhada do funcionamento de cada entidade do sistema e de suas interações.

Por meio do LucidChart foi criado o diagramas de caso de uso UML do sistema, devido à sua interface intuitiva e ampla gama de recursos específicos para UML, que facilitam a representação visual precisa dos requisitos funcionais do projeto. Sua capacidade de colaboração em tempo real permite uma colaboração eficiente entre os membros da equipe.

A plataforma Draw.io foi empregada para a construção do diagrama de redes do sistema proposto. O Draw.io é uma ferramenta online de diagramação que oferece uma interface intuitiva e uma ampla gama de recursos para criação de diagramas profissionais em diversas áreas, incluindo redes de sistemas. O uso da plataforma proporcionou uma abordagem eficaz para a criação do diagrama de redes do sistema estudado, permitindo uma representação visual clara e precisa de sua infraestrutura de rede.

Para criar o sistema desktop inicialmente utilizamos o Apex da Oracle, uma plataforma de desenvolvimento de aplicativos baseada em banco de dados. O Apex permite criar rapidamente aplicativos web escaláveis, aproveitando a potência do Oracle Database. Com sua interface de desenvolvimento intuitiva e ferramentas integradas, o Apex agiliza o processo de desenvolvimento e oferece flexibilidade para atender às necessidades específicas do sistema. Porém, demos continuidade na criação do sistema desktop utilizando a linguagem de programação Java Script através do programa Visual Studio Code e a plataforma Node.js.

O Visual Studio Code foi escolhido como ambiente de desenvolvimento pela sua interface intuitiva e pela ampla gama de extensões que otimizam a produtividade, enquanto o Node.js foi utilizado para construir o backend do sistema. A escolha do Node.js possibilitou a criação de uma estrutura de backend robusta, que oferece alta performance e escalabilidade, fundamentais para suportar o crescimento do sistema.

A combinação dessas ferramentas trouxe flexibilidade ao desenvolvimento, permitindo integrações com outras tecnologias e ampliando a capacidade de personalização do sistema para atender melhor às necessidades dos usuários. Dessa forma, a equipe conseguiu manter um fluxo de trabalho ágil e otimizado para desenvolver novas funcionalidades e aperfeiçoar a experiência final do sistema.