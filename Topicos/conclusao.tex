Para auxiliar o cumprimento dos Objetivos de Desenvolvimento Sustentável (ODS) estabelecidos pela Organização das Nações Unidas (ONU), o trabalho busca favorecer a terceira meta, relacionada à saúde e bem-estar, por meio do mapeamento de áreas de foco de dengue e que contenham pessoas infectadas pela covid-19. 

O sistema será desenvolvido para que agentes de saúde possam registrar localizações no mapa que são afetadas pelas doenças, a fim de que, por meio do HeatMap, o administrador tenha acesso em tempo real às áreas de risco e organize os processos de dedetização de acordo com o nível de impacto em cada região. Sendo assim, com este sistema, pretende-se reduzir o número de óbitos pela doença transmitida pelo Aedes aegypti e pelo coronavírus no Brasil, além de facilitar o trabalho da equipe de saúde. O sistema busca ser o mais fácil possível para os usuários, onde o agente apenas aponta o local afetado, o gestor, por sua vez, simplesmente administra os dados, e o dedetizador visualiza as rotas para fazer o tratamento dos locais.  

 Toda a atividade é feita através da aplicação para dispositivos móveis, conhecida como EPIDEMIX, inclusive a visualização do mapa de calor pelo agente de dedetização. Entretanto, também existe a versão desktop destinada aos supervisores. Vale ressaltar que o aplicativo também estará disponível para usuários comuns, ou seja, aqueles que não são agentes de saúde, para que possam visualizar o mapa com as localizações de risco caso tenham interesse em preservar a saúde e evitar a contaminação por estas doenças.

Ao analisar o Modelo de Negócios Canvas, observa-se que o projeto ainda demanda testes práticos em contextos reais com agentes de saúde e seus respectivos gestores, para obter relatórios e avaliações, a fim de que a viabilidade operacional e econômica do sistema possa ser aprofundada.

Ademais, pensando na evolução contínua do sistema, observa-se a necessidade de realizar mais estudos sobre as doenças através de pesquisas de campo com os próprios agentes e gestores de saúde, reconhecendo especificidades ou técnicas que possam contribuir para a melhoria e precisão dos dados oferecidos pelo projeto, em especial reconhecer os níveis de intensidade da região afetada. Além disso, tais pesquisas fornecerão retorno sobre a eficácia e usabilidade do sistema na identificação de áreas de risco de dengue e covid-19 no Brasil.

Com o intuito de criar um sistema eficiente, pretende-se utilizar o conceito do caminho mínimo para determinar as rotas mais eficientes para as equipes de dedetização por meio de um grafo, onde os nós representam as áreas afetadas e as arestas as rotas viáveis. Dessa forma, será possível planejar rotas de maneira mais eficaz, permitindo que mais locais sejam tratados em menos tempo, facilitando o trabalho da equipe e contribuindo para o combate aos focos do Aedes e ao controle de casos de covid-19. No que se refere a identificar os dados de forma intuitiva em mapas de calor, propõe-se utilizar o processo de homogenização. Neste processo, as áreas não marcadas com uma localização receberão um valor estimado para criar uma representação contínua dos dados, suavizando as variações e permitindo uma melhor visualização das regiões afetadas. \cite{25}. \cite{26}

Espera-se que ao incluir as particularidades apresentados ao projeto, o sistema seja um fator significante na redução de número de óbitos por dengue e covid-19 no território brasileiro, e ainda, possa ser uma forma de facilitar o processo de reconhecimento de áreas afetadas por estas doenças de forma prioritária e em tempo real, a fim de contribuir para o controle de tais regiões, assim como garantir a eficácia do aplicativo.Cabe citar que se espera que as novas tecnologias incluídas futuramente possam ser reconhecidas como fatores cruciais que promoverão a eficácia e flexibilidade do aplicativo.

Em suma, conclui-se que testes práticos e pesquisas de campo para comprovar a viabilidade e usabilidade ainda são necessários. Entretanto, o estudo apresentado até o momento mostra-se uma alternativa com potencial contributivo ao oferecer suporte aos agentes de saúde e gestores na identificação de áreas de risco de dengue e covid-19, haja vista que o sistema busca facilitar a iniciativa do processo de dedetização dessas regiões.